% Created 2018-09-26 mié 12:59
% Intended LaTeX compiler: pdflatex
\documentclass[11pt]{article}
\usepackage[utf8]{inputenc}
\usepackage[T1]{fontenc}
\usepackage{graphicx}
\usepackage{grffile}
\usepackage{longtable}
\usepackage{wrapfig}
\usepackage{rotating}
\usepackage[normalem]{ulem}
\usepackage{amsmath}
\usepackage{textcomp}
\usepackage{amssymb}
\usepackage{capt-of}
\usepackage{hyperref}
\usepackage[spanish]{babel}
\usepackage{fancyvrb}
\author{Dr. Miguel Carrillo Barajas \\
Estefanía Prieto Larios \\
Mauricio Esquivel Reyes \\
}
\date{}
\title{Sesión de laboratorio 05 \\
Lógica Computacional}
\hypersetup{
 pdfauthor={Dr. Miguel Carrillo Barajas \\
Estefanía Prieto Larios \\
Mauricio Esquivel Reyes \\
},
 pdftitle={Sesión de laboratorio 05 \\
Lógica Computacional},
 pdfkeywords={},
 pdfsubject={},
 pdfcreator={Emacs 25.3.2 (Org mode 9.1.14)}, 
 pdflang={Spanish}}
\begin{document}

\maketitle
\section{Solución de la práctica}
\label{sec:org808f0fe}
Se adjunta un archivo con el nombre p01.hs que contiene la solución de la primera práctica.
\section{Formulas de Horn}
\label{sec:org4fe9465}
La sintaxis de las formulas (proposicionales) de Horn, HORN, se define con notación BNF como sigue.

\[HORN                ::= <ClausulaDeHorn> | <ClausulaDeHorn> \land HORN.\]
\[<ClausulaDeHorn>    ::= (<Premisa> \rightarrow <Conclusion>)\]
\[<Premisa>           ::= <Atom> | (<Atom> \land <Premisa>)\]
\[<Conclusion>        ::= <Atom>\]
\[<Atom>              ::= \bot | \top | <Variable>\]
\[<Variable>          ::= v <Indice>\]
\[<Indice>            ::= i, i \in \mathbb{N}.\]

Una formula de Horn es una conjunción de clausulas de Horn.
Una clausula de Horn es una implicación cuya premisa es una conjuncion de 
átomos (\bot,\top,v\(_{\text{i}}\)) y cuya conclusión es un átomo.

\begin{verbatim}
-- Atomos para clausulas de Horn
data Hatom  =  Htop | Hbot | Hvar Indice deriving (Eq) 
-- Clausula de Horn: Premisa=[atomo1,...,atomoN] -> Conclusion=Atomo
data Fhorn  =   Himp [Hatom] Hatom                     
 -- Conjuncion de formulas de Horn 
              | Hand Fhorn Fhorn  deriving (Eq, Show)  
\end{verbatim}

\subsection{Atomos marcados}
\label{sec:org65143c1}
Nos regresa True si todos los atomos de una premisa estan marcados. False en otro caso
\begin{verbatim}
atomosMarcados :: [Hatom] -> [Hatom] -> Bool
atomosMarcados lm premisa = and [a `elem` lm | a <- premisa] 
\end{verbatim}

\subsection{Marca conclusiones}
\label{sec:orga991cc6}
Agrega las conclusiones de una lista de clausulas de Horn a una lista de atomos marcados.
\begin{verbatim}
marcaConclusiones :: [Hatom] -> [Fhorn] -> [Hatom]
marcaConclusiones lMarcados lcH = case lcH of
 [] -> lMarcados
 (Himp _ c):cHs -> marcaConclusiones (c:lMarcados) cHs
 _  -> error $ "marcaConclusiones: no es una clausula de Horn: "++show (head lcH)
\end{verbatim}

\subsection{Clausulas marcables}
\label{sec:org580102b}
Regresa la lista de clausulas de phi que tienen una conclusión marcable.
Def. \(f\) es una clausula con conclusión marcable si: \(f = p \rightarrow c\), los atomos de \(p\) están marcados, y \(c\) no está marcada.

\begin{verbatim}
clausulasHmarcables :: [Hatom] -> Fhorn -> [Fhorn]
clausulasHmarcables lMarcados phi = case phi of
 Himp premisa conclusion -> if (atomosMarcados lMarcados premisa) && 
(conclusion `notElem` lMarcados)
  then [phi]
  else []
 Hand f1 f2 -> clausulasHmarcables lMarcados f1 ++ clausulasHmarcables lMarcados f2
\end{verbatim}
\end{document}
