% Created 2018-08-16 Thu 13:09
% Intended LaTeX compiler: pdflatex
\documentclass[11pt]{article}
\usepackage[utf8]{inputenc}
\usepackage[T1]{fontenc}
\usepackage{graphicx}
\usepackage{grffile}
\usepackage{longtable}
\usepackage{wrapfig}
\usepackage{rotating}
\usepackage[normalem]{ulem}
\usepackage{amsmath}
\usepackage{textcomp}
\usepackage{amssymb}
\usepackage{capt-of}
\usepackage{hyperref}
\usepackage[spanish]{babel}
\usepackage{fancyvrb}
\author{Mauricio Esquivel Reyes}
\date{\today}
\title{Sesión de laboratorio 02}
\hypersetup{
 pdfauthor={Mauricio Esquivel Reyes},
 pdftitle={Sesión de laboratorio 02},
 pdfkeywords={},
 pdfsubject={},
 pdfcreator={Emacs 24.5.1 (Org mode 9.1.13)}, 
 pdflang={Spanish}}
\begin{document}

\maketitle
\tableofcontents


\section{Haskell}
\label{sec:orgda4522c}
Para refrescar lo aprendido, en esta ocasión veremos algunas funciones con los
tipos por defecto de haskell.
\subsection{Funciones}
\label{sec:org8f76f71}
\subsubsection{Potencia}
\label{sec:org1fe59d9}
\begin{verbatim}
pote :: Int -> Int -> Int
pote x 0 = 1
pote x y = x * (pote x (y-1))
\end{verbatim}
\subsubsection{Potencia B}
\label{sec:org0b40d48}
\begin{verbatim}
poteB :: Int -> Int -> Int
\end{verbatim}
\subsubsection{Toma primeros n elementos}
\label{sec:orge508b38}
\begin{verbatim}
toma :: Int -> [a] -> [a]
\end{verbatim}
\subsubsection{Mayores}
\label{sec:org3c686c0}
\begin{verbatim}
mayores :: Ord a => [a] -> a -> [a]
\end{verbatim}

\subsection{Tipos propios}
\label{sec:orgb68c14a}
Definiremos a los números naturales y algunas funciones de estos.
\subsubsection{Definición}
\label{sec:org28294b0}
\begin{verbatim}
data Natural = Cero | Suc Natural deriving (Eq, Show)
let tres = (Suc (Suc (Suc Cero)))
\end{verbatim}
\subsubsection{Suma}
\label{sec:org1587898}
\begin{verbatim}
suma :: Natural -> Natural -> Natural
\end{verbatim}
\subsubsection{Producto}
\label{sec:org9904d74}
\begin{verbatim}
prod :: Natural -> Natural -> Natural
\end{verbatim}
\subsubsection{Potencia}
\label{sec:org7add382}
\begin{verbatim}
potN :: Natural -> Natural -> Natural
\end{verbatim}
\subsubsection{Int a Natural}
\label{sec:org658a4f3}
\begin{verbatim}
int2Natural :: Natural -> Natural -> Natural
\end{verbatim}
\subsubsection{Natural a Int}
\label{sec:org2257adc}
\begin{verbatim}
natural2Int :: Natural -> Natural -> Natural
\end{verbatim}
\section{Lógica Proposicional}
\label{sec:orgddf4ce4}
\subsection{Sintaxis}
\label{sec:org8615a0e}
Esta es la sintaxis de la Lógica Proposicional que utilizaremos. 
\[PL ::= <ProposiciónAtómica> | \neg PL | (PL \land PL) | (PL \lor PL) | (PL \to PL) \]
\[<ProposiciónAtómica> ::= \top | \bot | <VariableProposicional>\]
\[<VariableProposicional> ::= v<Indice>\]
\[ <Indice> ::= [i | i \in \mathbb{N}]\]

\subsection{Definición en Haskell}
\label{sec:orga7844d5}
\begin{verbatim}
-- Tipo de dato indice
type Indice = Int

-- Tipo de dato fórmula
data PL = Top | Bot  | Var Indice
              | Oneg PL 
              | Oand PL PL | Oor PL PL 
              | Oimp PL PL deriving (Eq, Show)
\end{verbatim}

\subsection{Funciones}
\label{sec:org350b3ae}
\subsubsection{Elimina implicaciones}
\label{sec:org7538e94}
\begin{verbatim}
quitaImp :: PL -> PL
\end{verbatim}
\subsubsection{Forma Normal de Negación}
\label{sec:org8acc98a}
\begin{verbatim}
toNNF :: PL -> PL
\end{verbatim}
\subsubsection{Variables de una formula}
\label{sec:orgeb38236}
\begin{verbatim}
varsOf :: PL -> [PL]
\end{verbatim}
\end{document}
