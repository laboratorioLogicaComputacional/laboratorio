% Created 2018-08-10 Fri 14:16
% Intended LaTeX compiler: pdflatex
\documentclass[11pt]{article}
\usepackage[utf8]{inputenc}
\usepackage[T1]{fontenc}
\usepackage{graphicx}
\usepackage{grffile}
\usepackage{longtable}
\usepackage{wrapfig}
\usepackage{rotating}
\usepackage[normalem]{ulem}
\usepackage{amsmath}
\usepackage{textcomp}
\usepackage{amssymb}
\usepackage{capt-of}
\usepackage{hyperref}
\usepackage[spanish]{babel}
\usepackage{fancyvrb}
\author{Mauricio Esquivel Reyes}
\date{\today}
\title{Sesión de laboratorio 01}
\hypersetup{
 pdfauthor={Mauricio Esquivel Reyes},
 pdftitle={Sesión de laboratorio 01},
 pdfkeywords={},
 pdfsubject={},
 pdfcreator={Emacs 24.5.1 (Org mode 9.1.13)}, 
 pdflang={Spanish}}
\begin{document}

\maketitle
\tableofcontents


\section{Introducción}
\label{sec:org26b8e1d}
\subsection{¿Qué es Haskell?}
\label{sec:org3042a54}
Haskell es un lenguaje de programación multi-propósito puramente funcional
con semántica no estricta y tipado fuerte.
Haskell cuenta con inferencia de tipos, por lo cual no es necesario especificar
el tipo de las variables o las funciones.
La mayoría de los lenguajes son imperativos, en estos se especifica los
pasos que debe realizar el código. Los lenguajes funcionales trabajan de forma diferente.
En lugar de realizar acciones en secuencia, evalua expresiones.
\subsubsection{Ejemplo Java}
\label{sec:org0ff66a7}
\begin{verbatim}
int pot(int n, int m){
   int res = 1;
   for(int i = 0; i < m; i++){
      res *= n;
   }
}
\end{verbatim}
\subsubsection{Ejemplo Haskell}
\label{sec:org9a63f3a}
\begin{verbatim}
pot n 0 = 1
pot n m = n * (pot n (m-1)) 
\end{verbatim}
\subsection{Instalación}
\label{sec:org3755eff}
En la página de \url{https://www.haskell.org/downloads} podrán encontrar los 
paquetes de Haskell. Existen dos versiones Haskell Platform que incluye 
el manejador de paquetes, el compilador GHC y otras herramientas.
O también pueden descargar el compilador GHC que tiene el interpretre ghci.
\subsection{Tipos básicos}
\label{sec:orgd4f52b4}
Haskell tiene definido los siguientes tipos:
\begin{itemize}
\item Bool
\begin{itemize}
\item True
\item False
\item \&\&
\item ||
\item not
\end{itemize}
\item Char
\begin{itemize}
\item ++
\end{itemize}
\item Int
\item Integers
\item Float
\item Double
\begin{itemize}
\item +
\item -
\item /
\item *
\end{itemize}
\item Listas
\begin{itemize}
\item !!
\end{itemize}
\item Tuplas
\begin{itemize}
\item fst
\item snd
\end{itemize}
\end{itemize}

Para conocer el tipo de una expresión  en haskell solo se necesita hacer:
\begin{verbatim}
> :t 5
> :t "Hola mundo"
> :t False
> :t head
\end{verbatim}
\subsection{Variables de tipo}
\label{sec:org5e3696e}
Pero ¿qué es esa a? Los tipos que acabamos de ver empiezan con letra mayúscula.
La a es una variable de tipo, podemos pensarla como los genericos de otros lenguajes.
Estas variables de tipo son más poderosas que los genericos, ya que nos 
permiten escribir funciones muy generales mientras no dependan del comportamiento
especifico de los tipos. Estas funciones son llamadas polimórficas.
\subsection{Clases de tipo}
\label{sec:orgc9aa777}
Las clases de tipos son una especie de interfaz que define algún tipo de
comportamiento. Si un tipo es un miembro de una clase de tipos, significa 
que ese tipo soporta e implementa el comportamiento que define la clase de tipos.
Los podriamos ver como interfaces de Java.
\begin{verbatim}
> :t (==)
(==) :: Eq a => a -> a -> Bool
\end{verbatim}
Cualquier cosa antes del símbolo => es una restricción de clase.
Se lee: La función de igualdad toma dos parámetros que son del mismo tipo
y devuelve un Bool. El tipo de estos dos parámetros debe ser miembro de la
clase Eq.
\subsubsection{Básicas}
\label{sec:org3e48978}
\begin{itemize}
\item Eq
\item Ord
\item Show
\item Read
\item Enum
\end{itemize}
\subsection{Más allá}
\label{sec:orga77021f}
Esta es una introducción a haskell muy muy básica.
Para seguir aprendiendo hay bastante material 
en \url{https://www.haskell.org/documentation}

\section{Lógica}
\label{sec:orge31d1ae}
\subsection{Sintaxis}
\label{sec:org554b6df}
Esta es la sintaxis de la Lógica Proposicional que utilizaremos. 
\[PL ::= <ProposiciónAtómica> | \neg PL | (PL \land PL) | (PL \lor PL) | (PL \to PL) \]
\[<ProposiciónAtómica> ::= \top | \bot | <VariableProposicional>\]
\[<VariableProposicional> ::= v<Indice>\]
\[ <Indice> ::= [i | i \in \mathbb{N}]\]

\subsection{Definición en Haskell}
\label{sec:org1db401f}
\begin{verbatim}
-- Tipo de dato indice
type Indice = Int

-- Tipo de dato fórmula
data PL = Top | Bot 
              | Var Indice | Oneg PL 
              | Oand PL PL | Oor PL PL 
              | Oimp PL PL deriving (Eq, Show)
\end{verbatim}

\subsection{Funciones}
\label{sec:org66ef227}
\subsubsection{Número de operadores}
\label{sec:orge026b03}
\begin{verbatim}
numOp :: PL -> Int
numOp Top = 0
numOp Bot = 0
numOp (Var x) = 0
numOp (Oneg p) = numOp p + 1
numOp (Oand p q) = numOp p + numOp q + 1
numOp (Oor p q) = numOp p + numOp q + 1
numOp (Oimp p q) = numOp p + numOp q + 1
\end{verbatim}
\subsubsection{Elimina implicaciones}
\label{sec:org9fde8b0}
\begin{verbatim}
quitaImp :: PL -> PL
quitaImp Top = Top
quitaImp Bot = Bot
quitaImp (Var x) = Var x
quitaImp (Oneg p) = Oneg $ quitaImp p
quitaImp (Oand p q) = Oand (quitaImp p) (quitaImp q)
quitaImp (Oor p q) = Oor (quitaImp p) (quitaImp q)
quitaImp (Oimp p q) = Oor (Oneg $ quitaImp p) (quitaImp q)
\end{verbatim}
\subsubsection{Número de operadores binarios}
\label{sec:orgdfe1711}
\begin{verbatim}
numObin :: PL -> PL
numObin Top = 0
numObin Bot = 0
numObin (Var x) = 0
numObin (Oneg p) = numObin p
numObin (Oand p q) = numObin p + numObin q + 1
numObin (Oor p q) = numObin p + numObin q + 1
numObin (Oimp p q) = numObin p + numObin q + 1
\end{verbatim}
\subsubsection{Forma Normal Negativa}
\label{sec:org73a7fa9}
\begin{verbatim}
toNNF :: PL -> PL
toNNF p = toNNF $ quitaImp p where
  toNNF (Oneg (Oand p q)) = toNNF $ Oor (Oneg $ toNNF p) (Oneg $ toNNF q)
  toNNF (Oneg (Oor p q)) = toNNF $ Oand (Oneg $ toNNF p) (Oneg $ toNNF q)
  toNNF (Oneg (Oneg p)) = toNNF p
  toNNF (Oand p q) = Oand (toNNF p) (toNNF q)
  toNNF (Oor p q) = Oor (toNNF p) (toNNF q)
  toNNF p = p
\end{verbatim}
\end{document}
