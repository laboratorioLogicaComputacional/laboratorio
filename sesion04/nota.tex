% Created 2018-09-03 Mon 19:52
% Intended LaTeX compiler: pdflatex
\documentclass[11pt]{article}
\usepackage[utf8]{inputenc}
\usepackage[T1]{fontenc}
\usepackage{graphicx}
\usepackage{grffile}
\usepackage{longtable}
\usepackage{wrapfig}
\usepackage{rotating}
\usepackage[normalem]{ulem}
\usepackage{amsmath}
\usepackage{textcomp}
\usepackage{amssymb}
\usepackage{capt-of}
\usepackage{hyperref}
\usepackage[spanish]{babel}
\usepackage{fancyvrb}
\author{Dr. Miguel Carrillo Barajas \\
Estefanía Prieto Larios \\
Mauricio Esquivel Reyes \\
}
\date{}
\title{Sesión de laboratorio 04 \\
Lógica Computacional}
\hypersetup{
 pdfauthor={Dr. Miguel Carrillo Barajas \\
Estefanía Prieto Larios \\
Mauricio Esquivel Reyes \\
},
 pdftitle={Sesión de laboratorio 04 \\
Lógica Computacional},
 pdfkeywords={},
 pdfsubject={},
 pdfcreator={Emacs 24.5.1 (Org mode 9.1.2)}, 
 pdflang={Spanish}}
\begin{document}

\maketitle
\section{Solución práctica urgente 1}
\label{sec:org11e3858}
Veremos una solución de la práctica urgente 1
\subsection{Definición de Lógica Proposicional en Haskell.}
\label{sec:orgd4aa38d}
\begin{verbatim}
type Indice = Int
data PL =  Top | Bot | Var Indice -- Casos base
--Casos recursivos
| Oneg PL | Oand PL PL | Oor PL PL | Oimp PL PL deriving (Eq, Show)
\end{verbatim}
\subsection{Quita Implicaciones}
\label{sec:org99dd40e}
\begin{verbatim}
-- Función que quita implicaciones de una formula
quitaImp :: PL -> PL
quitaImp phi = case phi of
 Top -> Top
 Bot -> Bot
 Var x -> Var x
 Oneg x -> Oneg (quitaImp x)
 Oand x y -> Oand (quitaImp x) (quitaImp y)
 Oor x y -> Oor (quitaImp x) (quitaImp y)
 Oimp x y -> Oor (quitaImp (Oneg  x)) (quitaImp y)
\end{verbatim}
\subsection{Número de ocurrencias de variables}
\label{sec:org62bf225}
\begin{verbatim}
-- Función que cuenta las ocurrencias de variables en una formula
numOcurVar :: PL -> Int
numOcurVar phi = case phi of
 Top -> 0
 Bot -> 0
 Var _ -> 1
 Oneg alfa -> numOcurVar alfa
 Oand alfa beta -> numOcurVar alfa + numOcurVar beta
 Oor alfa beta -> numOcurVar alfa + numOcurVar beta
 Oimp  alfa beta -> numOcurVar alfa + numOcurVar beta
\end{verbatim}
\subsection{Compara número de ocurrencias}
\label{sec:org8ba5518}
\begin{verbatim}
-- Función que compara el número de ocurrencias de variables de una
-- formula y su forma normal de negación
compOcurVar :: PL -> Bool
compOcurVar phi = numOcurVar phi == (numOcurVar $ quitaImp phi)
\end{verbatim}
\section{\LaTeX{}}
\label{sec:org5f3d16c}
\subsection{Introducción}
\label{sec:org9f532b1}
\LaTeX{} es un sistema de composición de textos, orientados a la creación de documentos escritos que presentan una alta calidad tipográfica.
\LaTeX{} presupone una filosofía de trabajo diferente a la de los procesadores de texto habituales (conocidos como WYSIWYG, es decir, «lo que ves es lo que obtienes») y se basa en instrucciones. 
El modo en que \LaTeX{} interpreta la «forma» que debe tener el documento es mediante etiquetas. Por ejemplo, documentclass\{article\} le dice a \LaTeX{} que el documento que va a procesar es un artículo.
Después de haber creado un archivo de texto plano hay que compilarlo para que nos de el PDF.
Los reportes de las prácticas de laboratorio se entregarán en \LaTeX{}.
\subsection{Uso}
\label{sec:org2b4f564}
Pueden obtener \LaTeX{} en la siguiente dirección \url{https://www.latex-project.org/get/}
O pueden utilizar \url{https://www.overleaf.com} el cual te deja editar archivos, 
visualizarlos en el instante y  pueden editarlo entre varias personas.
\subsection{Ejemplo}
\label{sec:orgf2f1705}
\begin{verbatim}
\documentclass[11pt]{article}
\usepackage[utf8]{inputenc}
\usepackage[spanish]{babel}
\author{Hambone Fakenamington \\
Carla Mercedes Benz-Brown}
\title{Reporte Práctica XX \\
Lógica Computacional}
\date{}
\begin{document}
\maketitle
\section{Secciones y subsecciones}
\subsection Listas
\begin{itemize}
\item Uno
\item Two
\item Trois
\end{itemize}
\subsection{Enumeraciones}
\begin{enumerate}
\item Premier
\item Second
\item Tercero
\begin{itemize}
\item A
\item B
\item C
\end{itemize}
\end{enumerate}
\subsection{Formulas matemáticas}
Para escribirlas sobre la linea de texto es entre dos \$ \\
La formula $p$ esta en forma... \\
Para resaltarlas se escribe entre corchetes de la siguiente manera
\[ \prod_{i=a}^{b} f(i) \]
\end{document}
\end{verbatim}
\section{Implementación de la semántica}
\label{sec:org72dfb6f}
\subsection{Valuaciones}
\label{sec:org2ce28f4}
\begin{itemize}
\item Def. Decimos que \(\sigma\) es una valuación si \(\sigma: X \rightarrow {0,1}\) y \(\emptyset \neq X \subseteq Var\).
En particular, si \(\sigma: Var \rightarrow {0,1}\)
\end{itemize}
\begin{verbatim}
type Valuacion = Indice -> Bool
\end{verbatim}
\begin{itemize}
\item Def. Sean \(\sigma: Var \rightarrow {0,1}\) una valuación  y \(\phi \in PL\).
\end{itemize}
Definimos la relación  \(\sigma \models \phi\), usando recursión  sobre la estructura de \(\phi\)
\begin{verbatim}
satPL :: Valuacion -> PL -> Bool
satPL valor phi = case phi of
 Top -> True
 Bot -> False
 Var p -> (valor p)
 Oneg p -> Not(satPL p)
 Oand p q -> (satPL p) && (satPL q)
 Oor p q -> (satPL p) || (satPL q)
 Oimp p q -> Not(satPL p) || (satPL q)
\end{verbatim}
\subsection{Modelos}
\label{sec:org3422d76}
\begin{itemize}
\item Def. m es un modelo si \(m \subseteq Var\).
\end{itemize}
satMod m phi = True sii \(m \models \varphi\)
\begin{verbatim}
type Modelo = [Indice]
\end{verbatim}
\begin{itemize}
\item Def. Sean m un modelo y \(\phi \in PL\).
\end{itemize}
Definimos la relación  \(m \models \phi\), usando recursión  sobre la estructura de \(\phi\)
\begin{verbatim}
satMod :: Modelo -> PL -> Bool
satMod m phi = case phi of
 Top -> True
 Bot -> False
 Var p -> elem p m
 Oneg p -> Not(satMod p)
 Oand p q -> (satMod p) && (satMod q)
 Oor p q -> (satMod p) || (satMod q)
 Oimp p q -> Not(satMod p) || (satMod q)
\end{verbatim}
\subsection{Modelo a Valuacion}
\label{sec:org458a412}
\begin{verbatim}
modeloToValuacion :: Modelo -> Valuacion
modeloToValuacion m = sigma_m
    where
    sigma_m :: Valuacion
    sigma_m v = elem v m
\end{verbatim}
¿Qué resultado se espera de satPL \(\sigma\) \(\phi\) == satPL \(\sigma\) (toNNF \(\phi\))?
\subsection{Potencia de conjuntos}
\label{sec:org6d26c6f}
\begin{verbatim}
powerSet :: [t] -> [[t]]
powerSet l  = case l of
                   []   -> [[]]
                   x:xs -> powerXS ++ [x:w | w <- powerXS]
                            where
                            powerXS = powerSet xs 
\end{verbatim}
\section{Formas normales}
\label{sec:org0e76958}
\subsubsection{Negación}
\label{sec:org51637a6}
El objetivo de esta forma normal es obtener una fórmula equivalente a una fórmula dada sin
implicaciones, donde además los símbolos de negación solo afecten a fórmulas
atómicas.
\begin{verbatim}
-- Función que transforma una formula su forma normal de negación
-- Precondición: no debe tener implicaciones.
noImpNNF :: PL -> PL
noImpNNF phi = case phi of
  -- Casos base:
  Top -> phi
  Bot -> phi
  Var v -> Var v
  -- Casos recursivos:
  Oneg alfa -> case alfa of
    -- Casos bases (alfa)
    Top -> Bot
    Bot -> Top
    Var v -> Oneg (Var v)
    -- Casos recursivos (alfa)
    Oneg g -> noImpNNF g
    Oand g h -> noImpNNF (Oor (Oneg g) (Oneg h))
    Oor g h -> noImpNNF (Oand (Oneg g) (Oneg h))

  Oand alfa beta -> Oand (noImpNNF alfa) (noImpNNF beta)
  Oor alfa beta -> Oor (noImpNNF alfa) (noImpNNF beta)

-- Función que transforma una formula a su forma normal de negación.
-- Precondición: ninguna.
toNNF :: PL -> PL
toNNF = noImpNNF . quitaImp -- Composicion de funciones.
\end{verbatim}
\subsubsection{Conjunción}
\label{sec:org9fd0271}
La llamada forma normal de conjunción permite expresar cualquier fórmula proposicional como
una conjunción de disyunciones llamadas cláusulas.
\[CNF         ::= <Clausula> | (<Clausula> \land CNF).\]
\[<Clausula>  ::= \bot | <Literal>  | (<Literal> \lor <Clausula>)\]
\[<Literal>   ::= <Variable> | \neg \<Variable>\]
\[<Variable>  ::= v <Indice>\]
\[<Indice>    ::= i,  i \in \mathbb{N}\]

\begin{enumerate}
\item Es clausula
\label{sec:org139870c}
Función que nos indica si una fórmula tiene la forma de una clausula.
\begin{verbatim}
esClausula :: PL -> Bool
esClausula phi = case phi of
  Bot -> True
  Var _ -> True
  Oneg alfa -> case alfa of
    Var _ -> True
    _ -> False
  Oor alfa beta -> esClausula alfa && esClausula beta
  _ -> False
\end{verbatim}
\end{enumerate}
\end{document}
