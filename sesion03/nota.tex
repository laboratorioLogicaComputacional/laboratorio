% Created 2018-08-22 Wed 22:56
% Intended LaTeX compiler: pdflatex
\documentclass[11pt]{article}
\usepackage[utf8]{inputenc}
\usepackage[T1]{fontenc}
\usepackage{graphicx}
\usepackage{grffile}
\usepackage{longtable}
\usepackage{wrapfig}
\usepackage{rotating}
\usepackage[normalem]{ulem}
\usepackage{amsmath}
\usepackage{textcomp}
\usepackage{amssymb}
\usepackage{capt-of}
\usepackage{hyperref}
\usepackage[spanish]{babel}
\usepackage{fancyvrb}
\author{Mauricio Esquivel}
\date{\today}
\title{Sesión de laboratorio 03}
\hypersetup{
 pdfauthor={Mauricio Esquivel},
 pdftitle={Sesión de laboratorio 03},
 pdfkeywords={},
 pdfsubject={},
 pdfcreator={Emacs 24.5.1 (Org mode 9.1.2)}, 
 pdflang={Spanish}}
\begin{document}

\maketitle
\tableofcontents

\section{Lógica Proposicional}
\label{sec:org0b64e7f}
\subsection{Funciones}
\label{sec:orgaf07e5e}
Veremos algunas funciones que ya definimos para la lógica proposicional y 
realizaremos un par de nuevas funciones, haciendo recursión sobre la estructura.
\subsubsection{Elimina implicaciones}
\label{sec:orgfc7fb97}
\begin{verbatim}
quitaImp :: PL -> PL
quitaImp phi = case phi of
     Top -> Top
     Bot -> Bot
     Var x -> Var x
     Oneg x -> Oneg (quitaImp x)
     Oand x y -> Oand (quitaImp x) (quitaImp y)
     Oor x y -> Oor (quitaImp x) (quitaImp y)
     Oimp x y -> Oor (quitaImp (Oneg  x)) (quitaImp y)
\end{verbatim}
\subsubsection{Forma Normal de Negación}
\label{sec:orgdb48c3a}
\begin{verbatim}
toNNF :: PL -> PL
toNNF phi = case quitaImp phi of
     Oneg (Oand x y) -> toNNF $ Oor (Oneg $ toNNF x) (Oneg $ toNNF y)
     Oneg (Oor x y) -> toNNF $ Oand (Oneg $ toNNF x) (Oneg $ toNNF y)
     Oneg (Oneg x) -> toNNF x
     Oand x y -> Oand (toNNF x) (toNNF y)
     Oor x y -> Oor (toNNF x) (toNNF y)
     Oneg x -> Oneg (toNNF x)
     x -> x
\end{verbatim}
\subsubsection{Forma Normal de Negación 2}
\label{sec:orgda8afbe}
\begin{enumerate}
\item noImp2NNF
\label{sec:org82b4a30}
\begin{verbatim}
-- Precondición: phi no tiene operadores de implicación.
noImp2NNF :: PL -> PL
noImp2NNF phi = case phi of
       -- Casos base:
       Top -> phi
       Bot -> phi
       Var v -> Var v
       -- Casos recursivos:
       Oneg alfa -> case alfa of
              -- Casos bases (alfa)
              Top -> Bot
              Bot -> Top
              Var v -> Oneg (Var v)
              -- Casos recursivos (alfa)
              Oneg g -> noImp2NNF g
              Oand g h -> noImp2NNF (Oor (Oneg g) (Oneg h))
              Oor g h -> noImp2NNF (Oand (Oneg g) (Oneg h))
       Oand alfa beta -> Oand (noImp2NNF alfa) (noImp2NNF beta)
       Oor alfa beta -> Oor (noImp2NNF alfa) (noImp2NNF beta)
\end{verbatim}
\item toNNF2
\label{sec:org1cbac28}
\begin{verbatim}
-- Precondición: ninguna.
toNNF2 :: PL -> PL
toNNF2 = noImp2NNF . quitaImp -- Composicion de funciones.
\end{verbatim}
\end{enumerate}
\subsubsection{Disyunciones de una formula}
\label{sec:orge648da7}
\begin{verbatim}
disy :: PL -> [PL]
\end{verbatim}
\subsubsection{Número de disyunciones de una formula}
\label{sec:org09a0b8f}
\begin{verbatim}
numdisy :: PL -> Int
\end{verbatim}
\subsubsection{isInNFF}
\label{sec:orga427c01}
Función que nos indica si una formula esta en forma normal de negación.
\begin{verbatim}
isInNFF :: PL -> Bool
\end{verbatim}
\end{document}
