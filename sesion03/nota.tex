% Created 2018-08-23 Thu 17:31
% Intended LaTeX compiler: pdflatex
\documentclass[11pt]{article}
\usepackage[utf8]{inputenc}
\usepackage[T1]{fontenc}
\usepackage{graphicx}
\usepackage{grffile}
\usepackage{longtable}
\usepackage{wrapfig}
\usepackage{rotating}
\usepackage[normalem]{ulem}
\usepackage{amsmath}
\usepackage{textcomp}
\usepackage{amssymb}
\usepackage{capt-of}
\usepackage{hyperref}
\usepackage[spanish]{babel}
\usepackage{fancyvrb}
\author{Mauricio Esquivel Reyes}
\date{\today}
\title{Sesión de laboratorio 03}
\hypersetup{
 pdfauthor={Mauricio Esquivel Reyes},
 pdftitle={Sesión de laboratorio 03},
 pdfkeywords={},
 pdfsubject={},
 pdfcreator={Emacs 24.5.1 (Org mode 9.1.13)}, 
 pdflang={Spanish}}
\begin{document}

\maketitle
\tableofcontents

\section{Lógica Proposicional}
\label{sec:org6c8bd7a}
\subsection{Funciones}
\label{sec:org9190209}
Veremos algunas funciones que ya definimos para la lógica proposicional y 
realizaremos un par de nuevas funciones, haciendo recursión sobre la estructura.
\subsubsection{Elimina implicaciones}
\label{sec:org4ea8142}
\begin{verbatim}
quitaImp :: PL -> PL
quitaImp phi = case phi of
 Top -> Top
 Bot -> Bot
 Var x -> Var x
 Oneg x -> Oneg (quitaImp x)
 Oand x y -> Oand (quitaImp x) (quitaImp y)
 Oor x y -> Oor (quitaImp x) (quitaImp y)
 Oimp x y -> Oor (quitaImp (Oneg  x)) (quitaImp y)
\end{verbatim}
\subsubsection{Forma Normal de Negación}
\label{sec:orgd44b7c3}
\begin{verbatim}
toNNF :: PL -> PL
toNNF phi = case quitaImp phi of
 Oneg (Oand x y) -> toNNF $ Oor (Oneg $ toNNF x) (Oneg $ toNNF y)
 Oneg (Oor x y) -> toNNF $ Oand (Oneg $ toNNF x) (Oneg $ toNNF y)
 Oneg (Oneg x) -> toNNF x
 Oand x y -> Oand (toNNF x) (toNNF y)
 Oor x y -> Oor (toNNF x) (toNNF y)
 Oneg x -> Oneg (toNNF x)
 x -> x
\end{verbatim}
\subsubsection{Forma Normal de Negación 2}
\label{sec:orgd273ad0}
\begin{enumerate}
\item noImp2NNF
\label{sec:orgc1b016f}
\begin{verbatim}
-- Precondición: phi no tiene operadores de implicación.
noImp2NNF :: PL -> PL
noImp2NNF phi = case phi of
 -- Casos base:
 Top -> Top
 Bot -> Bot
 Var v -> Var v
 -- Casos recursivos:
 Oneg alfa -> case alfa of
  -- Casos bases (alfa)
  Top -> Bot
  Bot -> Top
  Var v -> Oneg (Var v)
  -- Casos recursivos (alfa)
   Oneg g -> noImp2NNF g
   Oand g h -> noImp2NNF (Oor (Oneg g) (Oneg h))
   Oor g h -> noImp2NNF (Oand (Oneg g) (Oneg h))
 Oand alfa beta -> Oand (noImp2NNF alfa) (noImp2NNF beta)
 Oor alfa beta -> Oor (noImp2NNF alfa) (noImp2NNF beta)
\end{verbatim}
\item toNNF2
\label{sec:orga4d350a}
\begin{verbatim}
-- Precondición: ninguna.
toNNF2 :: PL -> PL
toNNF2 = noImp2NNF . quitaImp -- Composicion de funciones.
\end{verbatim}
\end{enumerate}
\subsubsection{Disyunciones de una formula}
\label{sec:orgc9557c3}
\begin{verbatim}
disy :: PL -> [PL]
disy phi = case phi of 
 Top -> []
 Bot -> []
 Var v -> []
 Oneg alpha -> disy alpha 
 Oor alpha beta -> [Oor aplpha beta] ++ disy alpha ++ disy beta 
 Oand alpha beta -> disy alpha ++ disy beta
 Oimp alpha beta -> disy alpha ++ disy beta
\end{verbatim}
\subsubsection{Número de disyunciones de una formula}
\label{sec:org8b29bd7}
\begin{verbatim}
numdisy :: PL -> Int
numdisy phi -> case phi of
      Top -> 0
      Bot -> 0
      Var v -> 0
      Oneg alpha -> numdisy alpha
      Oor alpha beta -> 1 + numdisy alpha + numdisy beta
      Oand alpha beta -> numdisy alpha + numdisy beta
      Oimp alpha beta -> numdisy alpha + numdisy beta  
\end{verbatim}
\subsubsection{isInNFF}
\label{sec:orgb67a2cc}
Función que nos indica si una formula esta en forma normal de negación.
\begin{verbatim}
isInNFF :: PL -> Bool
isInNFF phi = case phi of
 Top-> True
 Bot-> True
 Var v -> True
 Oneg alpha -> case alpha of
  Var x -> True
  _ -> False 
 Oor alpha beta -> isInNFF alpha && isInNFF beta
 Oand alpha beta -> isInNFF alpha && isInNFF beta
 Oimp _ _ -> False
\end{verbatim}
\end{document}
