% Created 2018-10-01 lun 13:20
% Intended LaTeX compiler: pdflatex
\documentclass[11pt]{article}
\usepackage[utf8]{inputenc}
\usepackage[T1]{fontenc}
\usepackage{graphicx}
\usepackage{grffile}
\usepackage{longtable}
\usepackage{wrapfig}
\usepackage{rotating}
\usepackage[normalem]{ulem}
\usepackage{amsmath}
\usepackage{textcomp}
\usepackage{amssymb}
\usepackage{capt-of}
\usepackage{hyperref}
\usepackage[spanish]{babel}
\usepackage{fancyvrb}
\author{Dr. Miguel Carrillo Barajas \\
Estefanía Prieto Larios \\
Mauricio Esquivel Reyes \\
}
\date{}
\title{Sesión de laboratorio 06 \\
Lógica Computacional}
\hypersetup{
 pdfauthor={Dr. Miguel Carrillo Barajas \\
Estefanía Prieto Larios \\
Mauricio Esquivel Reyes \\
},
 pdftitle={Sesión de laboratorio 06 \\
Lógica Computacional},
 pdfkeywords={},
 pdfsubject={},
 pdfcreator={Emacs 25.3.2 (Org mode 9.1.14)}, 
 pdflang={Spanish}}
\begin{document}

\maketitle
\section{Formas normales}
\label{sec:org42b6a60}
\subsection{Literales de una formula}
\label{sec:orgd6aef05}
Conjunto de variables (solo indices) de una formula.
\begin{verbatim}
varsOf :: PL -> [Indice]
varsOf Top = []
varsOf Bot = []
varsOf (Var p) = [p]
varsOf (Oneg p) = case p of
  Var n -> [-n]
  _ -> varsOf p 
varsOf (Oand p q) = nub $ (varsOf p) ++ (varsOf q)
varsOf (Oor p q) = nub $ (varsOf p) ++ (varsOf q)
varsOf (Oimp p q) = nub $ (varsOf p) ++ (varsOf q)
\end{verbatim}
\subsection{Formula valida}
\label{sec:orgd740ba2}
Decide si \(\phi\) es valida.
\begin{verbatim}
enValPL :: PL -> Bool
enValPL phi = and[satMod y phi |y <- powerSet(varsOf(phi))]
\end{verbatim}
\subsection{Formula satisfacible}
\label{sec:orgf37a8ee}
Decide si \(\phi\) es satisfactible.
\begin{verbatim}
enSatPL :: PL -> Bool
enSatPl phi = or[satMod y phi | y <- powerSet(varsOf(phi))]
\end{verbatim}
\subsection{Lista de clausulas}
\label{sec:org7fa4778}
\begin{verbatim}
-- Nos da la lista de clausulas de una formula
-- Precondición: la formula debe estar en CNF
listClau :: PL -> [PL]
listClau phi = case phi of 
  Oand alpha betha -> listClau alpha ++ listClau betha 
  _-> [phi]
\end{verbatim}
\subsection{Lista de clausulas como lista de indices}
\label{sec:orga98602b}
\begin{verbatim}
-- Nos da las clasusulas de una formula en forma de listas de indices
-- Precondición: la formula debe estar en CNF
liteCNF :: PL -> [[Indice]]
liteCNF phi = if esCNF(phi) then 
  map varsOf (listClau phi)
 else
  error $ "No esta en CNF"
\end{verbatim}
\subsection{Lista de terminos}
\label{sec:orgc9dbfb1}
\begin{verbatim}
-- Nos da la lista de terminos de una formula
-- Precondición: la formula debe estar en DNF
listTerm :: PL -> [PL]
listTerm phi = case phi of 
  Oor alpha beta -> listTerm alpha ++ listTerm beta
  _ -> [phi]
\end{verbatim}
\subsection{Lista de terminos como lista de indices}
\label{sec:org8fae79e}
\begin{verbatim}
-- Nos da los terminos de una formula en forma de lista de indices
-- Precondición: la formula debe estar en DNF
liteDNF :: PL -> [[Indice]]
liteDNF phi = if esDNF(phi) then 
  map varsOf (listTerm phi)
 else
  error $ "No esta en DNF"
\end{verbatim}
\subsection{Indices complementarios}
\label{sec:org9f30760}
\begin{verbatim}
-- Nos indica si en una lista de indices existen dos complementarios
comple :: [Indice] -> Bool
comple lst = case lst of
  [] -> False
  x:xs -> if elem (-x) xs then True else comple xs
\end{verbatim}
\subsection{CNF valida}
\label{sec:org795663c}
\begin{verbatim}
-- Nos dice si una formula en CNF es valida
-- Precondición: la formula debe estar en CNF
valCNF :: PL -> Bool
valCNF phi = and (map comple(liteCNF(phi)))
\end{verbatim}
\subsection{DNF satisfacible}
\label{sec:org7c64f2f}
\begin{verbatim}
-- Nos dice si una formula en DNF es satisfacible
-- Precondición: la formula debe estar en DNF
satDNF phi = if esDNF phi then 
satDNF phi = or (map not (map comple (liteDNF(phi)))
\end{verbatim}
\section{Formulas de Horn}
\label{sec:org6b1790e}
La sintaxis de las formulas (proposicionales) de Horn, HORN, se define con notación BNF como sigue.

\[HORN                ::= <ClausulaDeHorn> | <ClausulaDeHorn> \land HORN.\]
\[<ClausulaDeHorn>    ::= (<Premisa> \rightarrow <Conclusion>)\]
\[<Premisa>           ::= <Atom> | (<Atom> \land <Premisa>)\]
\[<Conclusion>        ::= <Atom>\]
\[<Atom>              ::= \bot | \top | <Variable>\]
\[<Variable>          ::= v <Indice>\]
\[<Indice>            ::= i, i \in \mathbb{N}.\]

Una formula de Horn es una conjunción de clausulas de Horn.
Una clausula de Horn es una implicación cuya premisa es una conjuncion de 
átomos (\bot,\top,v\(_{\text{i}}\)) y cuya conclusión es un átomo.

\begin{verbatim}
-- Atomos para clausulas de Horn
data Hatom  =  Htop | Hbot | Hvar Indice deriving (Eq) 
-- Clausula de Horn: Premisa=[atomo1,...,atomoN] -> Conclusion=Atomo
data Fhorn  =   Himp [Hatom] Hatom                     
 -- Conjuncion de formulas de Horn 
              | Hand Fhorn Fhorn  deriving (Eq, Show)  
\end{verbatim}
\subsection{Atomos marcados}
\label{sec:org5897fbc}
Nos regresa True si todos los atomos de una premisa estan marcados. False en otro caso
\begin{verbatim}
atomosMarcados :: [Hatom] -> [Hatom] -> Bool
atomosMarcados lm premisa = and [a `elem` lm | a <- premisa] 
\end{verbatim}
\subsection{Marca conclusiones}
\label{sec:org11cfc9f}
Agrega las conclusiones de una lista de clausulas de Horn a una lista de atomos marcados.
\begin{verbatim}
marcaConclusiones :: [Hatom] -> [Fhorn] -> [Hatom]
marcaConclusiones lMarcados lcH = case lcH of
 [] -> lMarcados
 (Himp _ c):cHs -> marcaConclusiones (c:lMarcados) cHs
 _  -> error $ "marcaConclusiones: no es una clausula de Horn: "++show (head lcH)
\end{verbatim}
\subsection{Clausulas marcables}
\label{sec:org2cbac5d}
Regresa la lista de clausulas de \(\phi\) que tienen una conclusión marcable.
Def. \(f\) es una clausula con conclusión marcable si: \(f = p \rightarrow c\), los atomos de \(p\) están marcados, y \(c\) no está marcada.
\begin{verbatim}
clausulasHmarcables :: [Hatom] -> Fhorn -> [Fhorn]
clausulasHmarcables lMarcados phi = case phi of
 Himp premisa conclusion -> if (atomosMarcados lMarcados premisa) && 
(conclusion `notElem` lMarcados)
  then [phi]
  else []
 Hand f1 f2 -> clausulasHmarcables lMarcados f1 ++ clausulasHmarcables lMarcados f2
\end{verbatim}
\subsection{Marca formula de Horn}
\label{sec:org97c3066}
Mientras \(\phi\) tenga una clausula con conclusion marcable, marca las conclusiones de dichas clausulas.
Def. \(f\) es una clausula con conclusión marcable si: \(f = p \rightarrow c\), los atomos de \(p\) están marcados, y \(c\) no está marcada.
\begin{verbatim}
marcaFormulaHorn :: [Hatom] -> Fhorn -> [Hatom]
marcaFormulaHorn lMarcados phi = 
    if cHornMarcables == []
        then lMarcados
        else marcaFormulaHorn lMarcadosNew phi
    where
    cHornMarcables  = clausulasHmarcables lMarcados phi
    lMarcadosNew    = marcaConclusiones lMarcados cHornMarcables
\end{verbatim}
\subsection{Horn es satisfacible}
\label{sec:org7ccc68e}
\begin{verbatim}
enSatHorn :: Fhorn -> Bool
enSatHorn phi = if Hbot `elem` (marcaFormulaHorn [Htop] phi)
                    then False
                    else True
\end{verbatim}
\end{document}
